%%%%%%%%%%%%%%%%%%%%%%%%%%%%%%%%%%%%%%%%%%%%%%%%%%%%%%%%%%%%%%%%%%%%%%
% Template-TA-Fisika-ITS.v.1.0 2022.01
%%%%%%%%%%%%%%%%%%%%%%%%%%%%%%%%%%%%%%%%%%%%%%%%%%%%%%%%%%%%%%%%%%%%%%
% oleh: 1. Sasfan Arman Wella, BRIN (sasfan.a.wella@gmail.com)
%       2. Nadya Amalia, BRIN (amalianadd@gmail.com)
%       3. Fitrotul Millah, ITS (fitrotulmillah2000@gmail.com)
%       4. Nathasya Veronica, ITS (nathasyaveronica363@gmail.com) 
%
%  NB: Template ini bukanlah template resmi dari ITS, namun disusun
%      berdasarkan pedoman penulisan laporan TA Fisika ITS
%
% [ Bila ada fitur yang ditambahkan dalam template ini, silahkan
%   tambahkan nama Anda setelah nama pembuat template sebelumnya ]
%
%  Penjelasan singkat dari template ini 
%  dapat dilihat pada file README
% 
%%%%%%%%%%%%%%%%%%%%%%%%%%%%%%%%%%%%%%%%%%%%%%%%%%%%%%%%%%%%%%%%%%%%%%

%=====================================================================
% FORMAT DAN PENGATURAN UMUM:
%=====================================================================
\documentclass[a5paper, 11pt, final, openright, twoside]{report}
%%%%%%%%%%%%%%%%%%%%%%%%%%%%%%%%%%%%%%%%%%%%%%%%%%%%%%%%%%%%%%%%%%%%%%
% FORMAT DAN PENGATURAN UMUM:
%=====================================================================
\usepackage{multirow}
\usepackage[table]{xcolor}
\usepackage{tabularx,tikz}
\usepackage{caption}
\usepackage{float}
\usepackage{listings}
\usepackage{subcaption}
\usepackage{graphicx}
\usepackage{enumitem}
\usepackage{amsmath,amssymb,charter,color}
\usepackage[hidelinks]{hyperref}
\usepackage[utf8]{inputenc}
\usepackage{lipsum} % hanya untuk membuat teks dummy
\usepackage{wrapfig}
\usepackage{emptypage}
\usepackage{setspace}
\usepackage{rotating}

%======================================================================
%   Mengatur margin 
%======================================================================
\usepackage[top=25mm,left=25mm,right=20mm,bottom=25mm]{geometry}
    \linespread{1.3} % Gunakan 1.3 untuk spasi satu setengah, 
                     % Gunakan 1.6 untukspasi dua
    
%======================================================================
%   Mengatur bahasa yang digunakan
%======================================================================

\usepackage[bahasa]{babel}
    \selectlanguage{bahasa}

%======================================================================
%   Mengatur format BAB, SUBBAB, Floating Gambar dan Tabel
%======================================================================

\usepackage[explicit]{titlesec}
    \titleformat{\chapter}[display]
        {\normalfont\bfseries\centering}
        {\large\MakeUppercase{\chaptername}~\thechapter}
        {0.5em}{\large \MakeUppercase{#1}}
    \titleformat{\section}
        {\normalfont\normalsize\bfseries}{\thesection}
        {0.5em}{#1}
    \titleformat{\subsection}
        {\normalfont\normalsize\bfseries}{\thesubsection}
        {0.5em}{#1}
    
\usepackage{tocloft,etoolbox}
\apptocmd{\appendix}
    {\addtocontents{toc}{  
    \protect\addtolength\protect\cftchapnumwidth{-\mylength}
    \protect\renewcommand{\protect\cftchappresnum}{LAMPIRAN~}
    \protect\settowidth\mylength{
    \bfseries\protect\cftchappresnum\protect\cftchapaftersnum}
    \protect\addtolength\protect\cftchapnumwidth{\mylength}}}{}{}
\newlength\mylength

\renewcommand\cftchappresnum{BAB~}
\settowidth\mylength{\bfseries\cftchappresnum\cftchapaftersnum}
\addtolength\cftchapnumwidth{\mylength}
\renewcommand{\cftdotsep}{1}
\renewcommand{\cftchapleader}{\cftdotfill{\cftsecdotsep}}

\renewcommand\cftfigpresnum{Gambar~}
\settowidth\mylength{\cftfigpresnum\cftfigaftersnum}
\addtolength\cftfignumwidth{\mylength}

\renewcommand\cfttabpresnum{Tabel~}
\settowidth\mylength{\cfttabpresnum\cfttabaftersnum}
\addtolength\cfttabnumwidth{\mylength}

%======================================================================
%   Mengatur background pada cover
%======================================================================
\usepackage[pages=some]{background}
\backgroundsetup{
    scale=1,
    color=black,
    opacity=1.0,
    angle=0,
    contents={%
    \includegraphics[width=\paperwidth,height=\paperheight]
    {./halaman-depan/00-BGcover.jpg}
    }%
}

%======================================================================
%   Mengatur jenis font yang digunakan
%======================================================================
\usepackage{fontspec}
\setmainfont{Times New Roman} 
\setsansfont{Trebuchet MS} 
\setmonofont{Inconsolata}

%======================================================================
%   Untuk mendefinisikan halaman kosong
%======================================================================
\newcommand\halamanKosong{
    \newpage
    \vspace*{\fill}
    \begin{center}
        \textit{Halaman ini sengaja dikosongkan}
    \end{center}
    \vspace{\fill}
    \clearpage
}

%======================================================================
%   Mengatur agar paragraf pertama mempunyai indentasi
%======================================================================
\usepackage{indentfirst}
\setlength{\parindent}{2em} 

%======================================================================
%   Mengatur tentang jarak antara judul section dan paragraf
%======================================================================
\usepackage{titlesec}

\titlespacing\section{0pt}{12pt plus 4pt minus 2pt}{0pt plus 2pt minus 2pt}
\titlespacing\subsection{0pt}{12pt plus 4pt minus 2pt}{0pt plus 2pt minus 2pt}
\titlespacing\subsubsection{0pt}{12pt plus 4pt minus 2pt}{0pt plus 2pt minus 2pt}

%======================================================================
%   Mengatur header dan footer
%======================================================================
\usepackage{fancyhdr}
    \fancyhead{}
    \fancyfoot{}
    \setlength{\headheight}{15pt}
    \setlength{\headsep}{12pt}
    \setlength{\footskip}{30pt}
    \renewcommand{\headrulewidth}{0pt}
    \renewcommand{\footrulewidth}{0pt}
    
    \fancypagestyle{romawi}{%
    \setlength{\headheight}{15pt}
    \setlength{\headsep}{12pt}
    \setlength{\footskip}{30pt}
    \fancyfoot[CE,CO]{\thepage}
    \renewcommand{\headrulewidth}{0pt}
    \renewcommand{\footrulewidth}{0pt}
    }
    
    \fancypagestyle{konten}{%
    \setlength{\headheight}{15pt}
    \setlength{\headsep}{12pt}
    \setlength{\footskip}{30pt}
    \fancyhead[LE,RO]{\thepage}
    \fancyfoot[CE,CO]{}
    \renewcommand{\headrulewidth}{0pt}
    \renewcommand{\footrulewidth}{0pt}
    }
    
%======================================================================
%   Mengatur tentang format referensi
%======================================================================

\usepackage[square]{natbib}

%======================================================================
%   Mengatur template untuk menulis code
%======================================================================

\usepackage{listings}
\usepackage[framemethod=default]{mdframed}

\newmdenv[innerlinewidth=0.5pt, 
roundcorner=4pt,
linecolor=red,
innerleftmargin=6pt,
innerrightmargin=6pt,
innertopmargin=6pt,
innerbottommargin=6pt,
backgroundcolor=red,
]{mybox}
\newcommand{\unitcell}{\textit{unit cell }}
\newcommand{\exchange}{\textit{exchange }}
\newcommand{\corr}{\textit{correlation }}
\newcommand{\eh}[1]{{\color{red} EH:{#1}}}
\newcommand{\schro}{Schr{\"o}dinger }
\newcommand{\be}{\begin{equation}}
\newcommand{\ee}{\end{equation}}
\newcommand{\bea}{\begin{eqnarray}}
\newcommand{\eea}{\end{eqnarray}}
\newcommand{\HH}{{\cal H}}
\newcommand{\RR}{{\cal R}}
\newcommand{\p}{\partial}
\newcommand{\s}{\sigma}
\newcommand{\la}{\langle}
\newcommand{\ra}{\rangle}
\newcommand{\lla}{\left\langle}
\newcommand{\rra}{\right\rangle}
\newcommand{\lb}{\left[}
\newcommand{\rb}{\right]}
\newcommand{\lp}{\left(}
\newcommand{\rp}{\right)}
\newcommand{\Tr}{{\rm \, Tr\,}}
\newcommand{\bra}[1]{\la #1|}
\newcommand{\ket}[1]{| #1\ra}
\newcommand{\sgn}{{\rm sgn}\,}
\renewcommand{\Im}{{\rm Im}\,}
\renewcommand{\Re}{{\rm Re}\,}
\renewcommand{\vec}[1]{{\bf #1}}
\newcommand{\eps}{\varepsilon}
\renewcommand{\tilde}{\widetilde}
\def\nn{\nonumber\\}

\definecolor{codegreen}{rgb}{0,0.6,0}
\definecolor{codegray}{rgb}{0.5,0.5,0.5}
\definecolor{codepurple}{rgb}{0.58,0,0.82}
\definecolor{backcolour}{rgb}{0.95,0.95,0.92}

\lstdefinestyle{mystyle}{
    backgroundcolor=\color{backcolour},   
    commentstyle=\color{codegreen},
    keywordstyle=\color{magenta},
    numberstyle=\tiny\color{codegray},
    stringstyle=\color{codepurple},
    basicstyle=\ttfamily\small,
    breakatwhitespace=false,         
    breaklines=true,                 
    captionpos=b,                    
    keepspaces=true,                 
    numbers=left,                    
    numbersep=5pt,                  
    showspaces=false,                
    showstringspaces=false,
    showtabs=false,                  
    tabsize=2 
}

\lstset{style=mystyle}

%======================================================================
%   Agar sisi bagian kanan menjadi lebih rapih
%======================================================================
\emergencystretch=\maxdimen
\hyphenpenalty=10000
\hbadness=10000

%======================================================================
%   Mengatur jenis font untuk equations
%======================================================================
\usepackage{newtxmath}

%======================================================================
%   Mengatur format daftar isi, gambar, dan tabel
%======================================================================

\renewcommand{\cfttoctitlefont}{\hfil\large\bfseries\MakeUppercase}
\renewcommand{\cftloftitlefont}{\hfil\large\bfseries\MakeUppercase}
\renewcommand{\cftlottitlefont}{\hfil\large\bfseries\MakeUppercase}
\renewcommand{\cftsecleader}{\cftdotfill{\cftdotsep}}
\setlength\cftparskip{-2pt}
\setlength\cftbeforesecskip{2pt}
\setlength\cftbeforechapskip{2pt}

%%%%%%%%%%%%%%%%%%%%%%%%%%%%%%%%%%%%%%%%%%%%%%%%%%%%%%%%%%%%%%%%%%%%%%

\input{informasi}
%=====================================================================

%=====================================================================
% KODE RINGKAS UNTUK SIMBOL DAN SATUAN:
%=====================================================================
%%%%%%%%%%%%%%%%%%%%%%%%%%%%%%%%%%%%%%%%%%%%%%%%%%%%%%%%%%%%%%%%%%%%%%
% KODE RINGKAS UNTUK SIMBOL DAN SATUAN:
%=====================================================================
% Simbol Matematis
%---------------------------------------------------------------------

\def\imag{\mathrm{i}}
\def\euler{\mathrm{e}}
\def\diff{\mathrm{d}}

%---------------------------------------------------------------------
% Satuan
%---------------------------------------------------------------------

\def\unitangstrom{\,\textrm{\AA}}
\def\unitkg{\,\textrm{kg}}
\def\unitJ{\,\textrm{J}}
\def\unitev{\,\textrm{eV}}
\def\unitmev{\,\textrm{meV}}
\def\unitvolt{\,\textrm{V}}
\def\unitm{\,\textrm{m}}
\def\unitcm{\,\textrm{cm}}
\def\unitmm{\,\textrm{mm}}
\def\unitum{\,\mathrm{\mu}\textrm{m}}
\def\unitnm{\,\textrm{nm}}
\def\unitwn{\,\textrm{cm}^{-1}}
\def\unitsec{\,\textrm{s}}
\def\unitps{\,\textrm{ps}}
\def\unitfs{\,\textrm{fs}}
\def\unitdeg{^\circ}
\def\unitcelcius{\unitdeg\textrm{C}}
\def\unitpercent{\,\%}

%---------------------------------------------------------------------
% Singkatan
%---------------------------------------------------------------------

\def\its{\,Institut Teknologi Sepuluh Nopember}
\def\qe{\,\textsc{Quantum\:ESPRESSO}}
\def\btp{\,\textsc{BoltzTraP2}}
\def\schro{\,Schr\"{o}dinger}
\def\snse{\,SnSe}

%%%%%%%%%%%%%%%%%%%%%%%%%%%%%%%%%%%%%%%%%%%%%%%%%%%%%%%%%%%%%%%%%%%%%%


%=====================================================================


%=====================================================================
\begin{document}
%=====================================================================
    \pagestyle{romawi}
    \pagenumbering{roman}

    %======================================================================
%   Cover UTAMA
%======================================================================

    \thispagestyle{empty}

    \BgThispage
    
    \begin{flushleft}
        \includegraphics[width=25mm]{./halaman-depan/00-Logo-ITS.png}
    \end{flushleft}
    
    \vspace{24mm}
    
    \noindent {\textsf{\color{white}{%
    \textbf{\kodeTA} % Nama dan kode mata kuliah
    }}}
    
    \vspace{4mm}
    
    \begin{flushleft}
        \noindent {\Large\textsf{\color{white}
        {\MakeUppercase{\textbf{\judulTA}}}}}
    \end{flushleft}
    
    \vspace{4mm}
    
    {\noindent\textsf{\color{white}
    {\MakeUppercase{\textbf{\namaMahasiswa}\\[-2mm]
    \textbf{NRP. \noIndukMahasiswa}}}}}
    
    \vspace{3mm}
    
    {\noindent\textsf{\color{white}{%
        \textbf{Dosen Pembimbing}\\[-2mm]
        \textbf{\namaDosenPembimbingSatu}\\[-2mm]
        \textbf{\namaDosenPembimbingDua}
    }}}
    
    \vspace{3mm}
    
    {\noindent\textsf{\color{white}{%
        \textbf{\namaDepartemen}\\[-2mm]
        \textbf{\namaFakultas}\\[-2mm]
        \textbf{\namaUniversitas}\\[-2mm]
        \textbf{\namaKota}\\[-2mm]
        \textbf{\the\year}
    }}}


%======================================================================
%   Menambahkan halaman kosong setelah cover utama
%======================================================================

\cleardoublepage

%======================================================================
%   Halaman judul (Bahasa Indonesia)
%======================================================================

{
\setcounter{page}{1}
\addcontentsline{toc}{chapter}{HALAMAN JUDUL}
    
\begin{flushleft}
    \includegraphics[width=25mm]{./halaman-depan/00-Logo-ITS.png}
\end{flushleft}
    
\vspace{24mm}
    
\noindent {\textsf{\color{black}{%
\textbf{\kodeTA} % Nama dan kode mata kuliah
}}}
    
\vspace{4mm}
    
\begin{flushleft}
    \noindent {\Large\textsf{\color{black}
    {\MakeUppercase{\textbf{\judulTA}}}}}
\end{flushleft}
    
\vspace{4mm}
    
{\noindent\textsf{\color{black}
{\MakeUppercase{\textbf{\namaMahasiswa}\\[-2mm]
\textbf{NRP. \noIndukMahasiswa}}}}}
    
\vspace{3mm}
    
{\noindent\textsf{\color{black}{%
    \textbf{Dosen Pembimbing}\\[-2mm]
    \textbf{\namaDosenPembimbingSatu}\\[-2mm]
    \textbf{\namaDosenPembimbingDua}
}}}
    
\vspace{3mm}
    
{\noindent\textsf{\color{black}{%
    \textbf{\namaDepartemen}\\[-2mm]
    \textbf{\namaFakultas}\\[-2mm]
    \textbf{\namaUniversitas}\\[-2mm]
    \textbf{\namaKota}\\[-2mm]
    \textbf{\the\year}
}}}

%======================================================================
%   Mengambahkan halaman kosong setelah cover utama
%======================================================================

\halamanKosong

%======================================================================
%   Halaman judul (Bahasa Inggris)
%======================================================================

\begin{flushleft}
    \includegraphics[width=25mm]{./halaman-depan/00-Logo-ITS.png}
\end{flushleft}
    
\vspace{24mm}
    
\noindent {\textsf{\color{black}{%
\textbf{\kodeTAInggris} % Nama dan kode mata kuliah dalam bahasa Inggris
}}}
    
\vspace{4mm}
    
\begin{flushleft}
    \noindent {\Large\textsf{\color{black}
    {\MakeUppercase{\textbf{\judulTAInggris}}}}}
\end{flushleft}
    
\vspace{4mm}
    
{\noindent\textsf{\color{black}
{\MakeUppercase{\textbf{\namaMahasiswa}\\[-2mm]
\textbf{NRP. \noIndukMahasiswa}}}}}
    
\vspace{3mm}
    
{\noindent\textsf{\color{black}{%
    \textbf{Dosen Pembimbing}\\[-2mm]
    \textbf{\namaDosenPembimbingSatu}\\[-2mm]
    \textbf{\namaDosenPembimbingDua}
}}}
    
\vspace{3mm}
    
{\noindent\textsf{\color{black}{%
    \textbf{\namaDepartemenInggris}\\[-2mm]
    \textbf{\namaFakultasInggris}\\[-2mm]
    \textbf{\namaUniversitas}\\[-2mm]
    \textbf{\namaKota}\\[-2mm]
    \textbf{\the\year}
}}}

}
    \halamanKosong
    
%---------------------------------------------------------------------
%    HALAMAN DEPAN
%---------------------------------------------------------------------
    
    % LEMBAR PENGESAHAN
    %%%%%%%%%%%%%%%%%%%%%%%%%%%%%%%%%%%%%%%%%%%%%%%%%%%%%%%%%%%%%%%%%%%%%%
%
%   Secara umum, informasi yang dibutuhkan pada lembar pengesahan
%   ini diambil dari file "informasi.tex"
%
%%%%%%%%%%%%%%%%%%%%%%%%%%%%%%%%%%%%%%%%%%%%%%%%%%%%%%%%%%%%%%%%%%%%%%

\begin{center}
    {\large\textbf{LEMBAR PENGESAHAN}}
    \addcontentsline{toc}{chapter}{LEMBAR PENGESAHAN}
    \pagestyle{fancy}
\end{center}

%---------------------------------------------------------------------

\begin{center}
    
    {\large\MakeUppercase{\textbf{{\judulTA}}}}

    \vspace{5mm}
        
    {\large\textbf{TUGAS AKHIR}}

    \vspace{2mm}
    
    Diajukan untuk memenuhi syarat dalam \\[-2mm] 
    memperoleh gelar sarjana pada: \\[-2mm]
    %
    Program Sarjana \namaDepartemen \\[-2mm]
    \namaFakultas \\[-2mm]
    \namaUniversitas \\[-2mm]
    \namaKota \\[-2mm]

    \vspace{6mm}
    
    Oleh: \\
    
    {\textbf{\MakeUppercase{\namaMahasiswa}}}\\
    {\textbf{\MakeUppercase{\noIndukMahasiswa}}}\\

\end{center}

%---------------------------------------------------------------------

\begin{flushleft}
Disetujui oleh Dosen Pembimbing Tugas Akhir \\[2mm]

\begin{tabular}{ l c l c}
    Pembimbing I    & : & \namaDosenPembimbingSatu &
    ($\quad\quad\quad$) \\
                    &   & NIP. \nipDosenPembimbingSatu  & \\
    Pembimbing II   & : & \namaDosenPembimbingDua  &
    ($\quad\quad\quad$) \\
                    &   & NIP. \nipDosenPembimbingDua  & \\
\end{tabular}
\end{flushleft}

\vfill

\begin{flushright}
    \namaKota, \tanggalPengesahan
\end{flushright}

\vfill

%%%%%%%%%%%%%%%%%%%%%%%%%%%%%%%%%%%%%%%%%%%%%%%%%%%%%%%%%%%%%%%%%%%%%%
    \halamanKosong

    % ABSTRAK
    %%%%%%%%%%%%%%%%%%%%%%%%%%%%%%%%%%%%%%%%%%%%%%%%%%%%%%%%%%%%%%%%%%%%%%
%
%   Abstrak
%
%%%%%%%%%%%%%%%%%%%%%%%%%%%%%%%%%%%%%%%%%%%%%%%%%%%%%%%%%%%%%%%%%%%%%%

\begin{center}
    \addcontentsline{toc}{chapter}{ABSTRAK}
    \pagestyle{fancy}
\end{center}

%---------------------------------------------------------------------

\begin{center}
    {\textbf{\MakeUppercase{\judulTA}}}
\end{center}

\vspace{5mm}

\noindent \begin{tabular}{l c l}
    \textbf{Nama}       & \textbf{:} & \textbf{\namaMahasiswa}  \\[-1mm]
    \textbf{NRP}        & \textbf{:} & \textbf{\noIndukMahasiswa}  \\[-1mm]
    \textbf{Departemen} & \textbf{:} & \textbf{\namaDepartemen}  \\[-1mm]
    \textbf{Pembimbing} & \textbf{:} & \textbf{1. \namaDosenPembimbingSatu}  \\[-1mm]
                        &            & \textbf{2. \namaDosenPembimbingDua}
\end{tabular}

%---------------------------------------------------------------------

\vspace{5mm}

\begin{center}
    \noindent {\textbf{{Abstrak}}}
\end{center}

%---------------------------------------------------------------------

% Catatan: Gunakan \singlespacing di tiap awal paragraf

{\singlespacing\indent%
Ini adalah contoh dokumen Tugas Akhir yang dibuat dengan menggunakan \textit{template} \LaTeX{} dengan format yang telah disesuaikan dengan aturan penulisan Tugas Akhir yang berlaku di Departemen Fisika, \its{} (ITS). \textit{Template} ini dibuat dengan tujuan untuk memudahkan mahasiswa/(i) dalam melakukan penyusunan Tugas Akhir sekaligus untuk dapat digunakan sebagai \textit{template} yang berlaku umum, dengan beberapa penyesuaian untuk Tugas Akhir di Departemen lain di ITS. \textit{File} Tugas Akhir dalam format \texttt{$^*$.pdf} akan dapat dihasilkan dengan mengkompilasi \texttt{main.tex} menggunakan \textit{compiler} Lua\LaTeX.
}

%---------------------------------------------------------------------

\vspace{5mm}

\noindent \textbf{Kata kunci:} \textit{katakunci-1, katakunci-2, katakunci-3, katakunci-4} % Kata kunci dalam bahasa Indonesia

%%%%%%%%%%%%%%%%%%%%%%%%%%%%%%%%%%%%%%%%%%%%%%%%%%%%%%%%%%%%%%%%%%%%%%
    \halamanKosong
    
    %%%%%%%%%%%%%%%%%%%%%%%%%%%%%%%%%%%%%%%%%%%%%%%%%%%%%%%%%%%%%%%%%%%%%%
%
%   Abstract
%
%%%%%%%%%%%%%%%%%%%%%%%%%%%%%%%%%%%%%%%%%%%%%%%%%%%%%%%%%%%%%%%%%%%%%%

\begin{center}
    \addcontentsline{toc}{chapter}{\textit{ABSTRACT}}
    \pagestyle{fancy}
\end{center}

%---------------------------------------------------------------------

\begin{center}
    {\textbf{\MakeUppercase{\judulTAInggris}}}
\end{center}

\vspace{5mm}

\noindent \begin{tabular}{l c l}
    \textbf{Name}       & \textbf{:} & \textbf{\namaMahasiswa}  \\[-1mm]
    \textbf{NRP}        & \textbf{:} & \textbf{\noIndukMahasiswa}  \\[-1mm]
    \textbf{Department} & \textbf{:} & \textbf{\namaDepartemenInggris}  \\[-1mm]
    \textbf{Supervisors}& \textbf{:} & \textbf{1. \namaDosenPembimbingSatu}  \\[-1mm]
                        &            & \textbf{2. \namaDosenPembimbingDua}
\end{tabular}

%---------------------------------------------------------------------

\vspace{5mm}

\begin{center}
    \noindent {\textbf{{\textit{Abstract}}}}
\end{center}

%---------------------------------------------------------------------

% Catatan: Gunakan \singlespacing di tiap awal paragraf

{\singlespacing\indent% 
Abstrak Tugas Akhir dalam bahasa Inggris dituliskan di sini.
}

%---------------------------------------------------------------------

\vspace{5mm}

\noindent \textbf{Keywords:} \textit{katakunci-1, katakunci-2, katakunci-3, katakunci-4} % Kata kunci dalam bahasa Inggris

%%%%%%%%%%%%%%%%%%%%%%%%%%%%%%%%%%%%%%%%%%%%%%%%%%%%%%%%%%%%%%%%%%%%%%
    \halamanKosong
    
    % KATA PENGANTAR 
    %%%%%%%%%%%%%%%%%%%%%%%%%%%%%%%%%%%%%%%%%%%%%%%%%%%%%%%%%%%%%%%%%%%%%%

\begin{center}
    {\textbf{KATA PENGANTAR}}
    \addcontentsline{toc}{chapter}{KATA PENGANTAR}
    \pagestyle{fancy}
\end{center}

Kata pengantar untuk Tugas Akhir yang berjudul “{\MakeUppercase{\textbf{\judulTA}}}”, untuk memenuhi persyaratan menyelesaikan pendidikan strata satu (S1) di \namaDepartemen, \namaFakultas, \namaUniversitas. Ucapan terima kasih kepada berbagai pihak dapat diungkapkan pada bagian ini. 

\vspace{6mm}

\begin{flushright}

\namaKota, \tanggalPengesahan

\vspace{6mm}

Penulis

\end{flushright}

\newpage
    %\halamanKosong 

    % DAFTAR ISI  
        \addcontentsline{toc}{chapter}{DAFTAR ISI}
    \tableofcontents
    \newpage
    %\halamanKosong 
    
    % DAFTAR GAMBAR
        \addcontentsline{toc}{chapter}{DAFTAR GAMBAR}
    \listoffigures
    \newpage
    %\halamanKosong
    
    % DAFTAR TABEL
        \addcontentsline{toc}{chapter}{DAFTAR TABEL}
    \listoftables
    \newpage
    %\halamanKosong
    
%---------------------------------------------------------------------
%   KONTEN
%   Silahkan buat file bab isian di folder "konten"
%---------------------------------------------------------------------

    \input{./konten/bab01}
    %%%%%%%%%%%%%%%%%%%%%%%%%%%%%%%%%%%%%%%%%%%%%%%%%%%%%%%%%%%%%%%%%%%%%%
% BAB TINJAUAN PUSTAKA
%=====================================================================
\renewcommand{\thechapter}{\Roman{chapter}}
\addtocontents{toc}{\vskip10pt}
\chapter{TINJAUAN PUSTAKA}
\renewcommand{\thechapter}{\arabic{chapter}}
%---------------------------------------------------------------------

%=====================================================================
\section{Melakukan Sitasi }
%=====================================================================

Referensi berupa artikel dari jurnal ilmiah yang digunakan dalam dokumen Tugas Akhir ini, dan datanya telah dimasukkan dalam \texttt{pustaka.bib}, dapat disitasi dengan cara \citet{refJurnal} atau \citep{refJurnal}. Referensi yang digunakan juga dapat berupa artikel dari \textit{proceedings} ilmiah dan buku \citep{refProceedings,refBuku}. Referensi berupa laman internet dapat langsung dituliskan sebagai \url{https://phys.org/news/2021-11-thermoelectric-crystal-high.html} atau dengan melakukan sitasi \citep{refInternet}.

%=====================================================================
\section{Menuliskan Persamaan Matematika}
%=====================================================================

Suatu persamaan dapat dituliskan seperti contoh persamaan \textit{figure of merit} termoelektrik berikut:

\begin{equation}
    ZT = \dfrac{S^2 \sigma}{\kappa} T,
    \label{eq:persamaan1}
\end{equation}

\noindent di mana, $S$ adalah koefisien Seebeck dengan satuan V.K$^{-1}$, $\sigma$ adalah konduktivitas listrik dengan satuan S/m, $\kappa$ adalah konduktivitas panas dengan satuan W.m$^{-1}$.K$^{-1}$, dan $T$ adalah suhu termoelektrik dengan satuan K. 

Persamaan~\ref{eq:persamaan1} adalah contoh persamaan matematika dengan penomoran. Nomor persamaan pada \textit{template} \LaTeX{} Tugas Akhir ini telah diatur untuk diurutkan berdasarkan urutan kemunculan (posisi) persamaan tersebut dalam \textit{file} \texttt{$^*$.tex} dalam bagian \texttt{konten}. Untuk menuliskan persamaan tanpa penomoran dapat digunakan:

\begin{displaymath}
    \kappa = \kappa_{\rm e} + \kappa_{\rm l}
\end{displaymath}

\noindent Berbeda dengan Persamaan~\ref{eq:persamaan1}, persamaan di atas tidak memiliki nomor.


%=====================================================================
\section{Membuat Tabel}
%=====================================================================

Pada dasarnya ada berbagai tipe tabel. Tabel~\ref{tab:tabel1} adalah contoh tabel yang paling umum digunakan, yang terdiri atas 3 kolom dan 4 baris.

\begin{table}[ht] % [h] menyatakan posisi. h: here, t: top, b: bottom
	\centering
	\caption{Contoh tabel yang paling umum digunakan}
	\label{tab:tabel1}
	\begin{tabular}{c c c} % c: center, l: left (rata kiri), r: right (rata kanan)
		\hline
		Kolom-1 baris-1   & Kolom-2 baris-1   & Kolom-3 baris-1 \\
		\hline
		Kolom-1 baris-2   & Kolom-2 baris-2   & Kolom-3 baris-2 \\
		Kolom-1 baris-3   & Kolom-2 baris-3   & Kolom-3 baris-3 \\
		Kolom-1 baris-4   & Kolom-2 baris-4   & Kolom-3 baris-4 \\
		\hline
	\end{tabular}
\end{table}

\begin{sidewaystable}[h!] % [h] menyatakan posisi. h: here, t: top, b: bottom
	\centering
	\caption{Contoh tabel dengan \textit{multicolumn} dan \textit{multirow}}
	\label{tab:multitab}
	\begin{tabular}{l c c r} % c: center, l: left (rata kiri), r: right (rata kanan)
		\hline
		\multirow{2}{*}{Kol-1 baris-1/2}  & \multicolumn{2}{c}{Kol-2/3 baris 1}  & \multirow{2}{*}{Kol-4 baris-1/2} \\
		                                  & Kol-2 baris-2                        & Kol-3 baris-2   & \\
		\hline
		Kol-1 baris-3  & Kol-2 baris-3    & Kol-3 baris-3   & Kol-4 baris-3 \\
		Kol-1 baris-4  & Kol-2 baris-4    & Kol-3 baris-4   & \multirow{2}{*}{Kol-4 baris-4/5} \\
		Kol-1 baris-5  & $^*$Kol-2 baris-5    & $^{**}$Kol-3 baris-5   & \\		                                  
		\hline
		\multicolumn{4}{r}{$^*$\citealt{refJurnal}} \\ % Contoh sitasi catatan kaki
		\multicolumn{4}{r}{$^{**}$\citealt{refProceedings}} % Contoh sitasi catatan kaki
	\end{tabular}
\end{sidewaystable}

Sementara itu, Tabel~\ref{tab:multitab} adalah contoh tabel \textit{landscape} yang menggunakan \textit{multicolumn} dan \textit{multirow}, yang pada dasarnya terdiri dari 4 kolom dan 5 baris.

\newpage % Untuk melanjutkan bagian di bawah ini pada halaman baru
%---------------------------------------------------------------------
\subsection{Memasukkan Gambar dan Kode}
%---------------------------------------------------------------------

\textit{File} gambar perlu diunggah terlebih dahulu ke dalam folder \texttt{gambar}. Keterangan Gambar~\ref{fig:gambar1} dapat dituliskan pada \texttt{caption}.

\begin{figure}[h] % [h] menyatakan posisi. h: here, t: top, b: bottom
    \centering
    \includegraphics[width=5cm]{./gambar/contoh.png}
    \caption{Keterangan gambar dapat dituliskan di sini.}
    \label{fig:gambar1}
\end{figure}

Apabila mahasiswa/(i) perlu untuk menampilkan kode, \textit{script} program atau sejenisnya, contoh di bawah ini dapat digunakan sebagai acuan.

{\color{blue}
\begin{lstlisting}
sudo apt install build-essential g++ gfortran 
sudo apt install libblas-dev liblapack-dev 
    libopenmpi-dev libscalapack-mpi-dev 
\end{lstlisting}
}


%---------------------------------------------------------------------
\subsection{Memuat kode ringkas dari simbol, satuan, dan singkatan}
%---------------------------------------------------------------------

Untuk memuat kode ringkas dari simbol, satuan, dan singkatan yang telah dinyatakan dalam \texttt{kodeUnit.tex}, sebagai contoh, dapat dilakukan dengan \schro{} atau \its. Kode tersebut dapat diubah dan kode lain dapat ditambahkan pada \texttt{kodeUnit.tex}.

\newpage
%---------------------------------------------------------------------
\subsection{Daftar kode untuk simbol matematika dan Yunani}
%---------------------------------------------------------------------

\begin{tabular}{l l}
$\leq$ &  $\backslash$leq \\
$\geq$ &  $\backslash$geq \\
$\neq$ &  $\backslash$neq \\
$\nleq$ &  $\backslash$nleq \\
$\ngeq$ &  $\backslash$ngeq \\
$\cong$ &  $\backslash$cong \\
$\equiv$ &  $\backslash$equiv \\
$\sim$ &  $\backslash$sim \\
$\approx$ &  $\backslash$approx \\
$\times$ &  $\backslash$times \\
$\cdot $ &  $\backslash$cdot \\
$\ast $ &  $\backslash$ast \\
$\div$ &  $\backslash$div \\
$\pm$ &  $\backslash$pm \\
$\mp$ &  $\backslash$mp \\
$\oplus$ &  $\backslash$oplus \\
$\otimes$ &  $\backslash$otimes \\
$\propto $ &  $\backslash$propto \\
$\infty$ & $\backslash$infty \\
$\because$ &  $\backslash$because \\
$\therefore$ &  $\backslash$therefore \\
\end{tabular}
\hspace*{1ex}
\begin{tabular}{ll}
$\in$ &  $\backslash$in \\
$\subset $ &  $\backslash$subset \\
$\subseteq $ &  $\backslash$subseteq \\
$\varnothing $ &  $\backslash$varnothing  \\
$\cap $ &  $\backslash$cap \\
$\cup $ &  $\backslash$cup \\
$\Rightarrow$ &  $\backslash$Rightarrow \\
$\rightarrow$ &  $\backslash$rightarrow \\
$\partial$ &  $\backslash$partial \\
$90^\circ$ &  90$^\wedge\backslash$circ \\
$\parallel$ &  $\backslash$parallel \\
$\bot$ &  $\backslash$bot \\
$\triangle$ &  $\backslash$triangle \\
$\nabla$ &   $\backslash$nabla \\
$\angle$ &  $\backslash$angle \\
$\Pi$ &  $\backslash$Pi \\
$\Theta$ &  $\backslash$Theta \\
$\Gamma$ &  $\backslash$Gamma \\
$\Delta$ &  $\backslash$Delta \\
$\Omega$ &  $\backslash$Omega \\
$\Sigma$ &  $\backslash$Sigma \\
\end{tabular}
\hspace*{1ex}
\begin{tabular}{l l}
\& & $\backslash$\& \\
\% & $\backslash$\% \\
$\alpha$ &  $\backslash$alpha \\
$\beta$ &  $\backslash$beta \\
$\epsilon$ &  $\backslash$epsilon \\
$\zeta$ &  $\backslash$zeta \\
$\eta$ &  $\backslash$eta \\
$\kappa$ &  $\backslash$kappa \\
$\lambda$ &  $\backslash$lambda \\
$\mu$ &  $\backslash$mu \\
$\xi$ &  $\backslash$xi \\
$\rho$ &  $\backslash$rho \\
$\tau$ &  $\backslash$tau \\
$\phi$ &  $\backslash$phi \\
$\psi$ &  $\backslash$psi \\
$\pi$ &  $\backslash$pi \\
$\theta$ &  $\backslash$theta \\
$\gamma$ &  $\backslash$gamma\\
$\delta$ &  $\backslash$delta \\
$\omega$ &  $\backslash$omega \\
$\sigma$ &  $\backslash$sigma \\
\end{tabular}
 

\newpage

    %%%%%%%%%%%%%%%%%%%%%%%%%%%%%%%%%%%%%%%%%%%%%%%%%%%%%%%%%%%%%%%%%%%%%%
% BAB METODOLOGI
%=====================================================================
\renewcommand{\thechapter}{\Roman{chapter}}
\addtocontents{toc}{\vskip10pt}
\chapter{METODOLOGI}
\renewcommand{\thechapter}{\arabic{chapter}}
%---------------------------------------------------------------------

%=====================================================================
\section{Diagram Alir Penelitian}
%=====================================================================

Dalam penelitian, untuk diperoleh hasil yang baik maka dalam melakukan penelitian harus melalui tahapan-tahapan secara urut dan runtut. Tahapan-tahapan tersebut digambarkan dalam bentuk diagram alir yang ditunjukkan pada Gambar~\ref{fig:diagramAlirPenelitian}.

\begin{figure}
    \centering
    \includegraphics[width=7cm]{./gambar/flowchart.png}
    \caption{Diagram alir penelitian.}
    \label{fig:diagramAlirPenelitian}
\end{figure}


%=====================================================================
\section{Jenis dan Desain Penelitian}
%=====================================================================

...


%=====================================================================
\section{Lokasi dan Waktu Penelitian}
%=====================================================================

...


%=====================================================================
\section{Prosedur Penelitian}
%=====================================================================

%---------------------------------------------------------------------
\subsection{Perangkat Penelitian}
%---------------------------------------------------------------------

%Lorem ipsum.
%\setlist{nolistsep}
%\begin{enumerate}[noitemsep]
%    \item Lorem ipsum.
%    \item ...
%\end{enumerate}


%---------------------------------------------------------------------
\subsection{Langkah Kerja}
%---------------------------------------------------------------------

\vspace{3mm}

\subsubsection{Subsubbagian 1}

%Lorem ipsum.
%\setlist{nolistsep}
%\begin{enumerate}[noitemsep]
%    \item Lorem ipsum.
%    \item ...
%\end{enumerate}

\subsubsection{Subsubbagian 2}

%Lorem ipsum.
%\setlist{nolistsep}
%\begin{enumerate}[noitemsep]
%    \item Lorem ipsum.
%    \item ...
%\end{enumerate}

\subsubsection{Subsubbagian 3}

%Lorem ipsum.
%\setlist{nolistsep}
%\begin{enumerate}[noitemsep]
%    \item Lorem ipsum.
%    \item ...
%\end{enumerate}
    %%%%%%%%%%%%%%%%%%%%%%%%%%%%%%%%%%%%%%%%%%%%%%%%%%%%%%%%%%%%%%%%%%%%%%
%%%%%%%%%%%%%%%%%%%%%%%%%%%%%%%%%%%%%%%%%%%%%%%%%%%%%%%%%%%%%%%%%%%%%%
% BAB METODOLOGI PENELITIAN:
%=====================================================================
\renewcommand{\thechapter}{\Roman{chapter}}
\addtocontents{toc}{\vskip10pt}
\chapter{HASIL DAN PEMBAHASAN}
\renewcommand{\thechapter}{\arabic{chapter}}
%---------------------------------------------------------------------

%=====================================================================
\section{Bagian 1}
%=====================================================================

...
%Lorem ipsum.
%\setlist{nolistsep}
%\begin{enumerate}[noitemsep]
%    \item Lorem ipsum.
%    \item ...
%\end{enumerate}


%---------------------------------------------------------------------
\subsection{Subbagian 1}
%---------------------------------------------------------------------

...
%Lorem ipsum.
%\setlist{nolistsep}
%\begin{enumerate}[noitemsep]
%    \item Lorem ipsum.
%    \item ...
%\end{enumerate}


%---------------------------------------------------------------------
\subsection{Subbagian 2}
%---------------------------------------------------------------------

...
%Lorem ipsum.
%\setlist{nolistsep}
%\begin{enumerate}[noitemsep]
%    \item Lorem ipsum.
%    \item ...
%\end{enumerate}


%=====================================================================
\section{Bagian 2}
%=====================================================================

...
%Lorem ipsum.
%\setlist{nolistsep}
%\begin{enumerate}[noitemsep]
%    \item Lorem ipsum.
%    \item ...
%\end{enumerate}



    %%%%%%%%%%%%%%%%%%%%%%%%%%%%%%%%%%%%%%%%%%%%%%%%%%%%%%%%%%%%%%%%%%%%%%
%%%%%%%%%%%%%%%%%%%%%%%%%%%%%%%%%%%%%%%%%%%%%%%%%%%%%%%%%%%%%%%%%%%%%%
% BAB PENUTUP
%=====================================================================
\renewcommand{\thechapter}{\Roman{chapter}}
\addtocontents{toc}{\vskip10pt}
\chapter{PENUTUP}
\renewcommand{\thechapter}{\arabic{chapter}}
%---------------------------------------------------------------------

%=====================================================================
\section{Kesimpulan}
%=====================================================================

Ini adalah contoh tulisan untuk bagian kesimpulan yang apabila dirincikan dapat dituliskan sebagai berikut:

\begin{enumerate}
    \item Saran 1. 
    \item Saran 2.
    \item Saran 3.
\end{enumerate}

%=====================================================================
\section{Saran}
%=====================================================================
\begin{enumerate}
    \item Kesimpulan 1.
    \item Kesimpulan 2.
\end{enumerate}


    
%---------------------------------------------------------------------
%   DAFTAR PUSTAKA
%---------------------------------------------------------------------

    \addtocontents{toc}{\vskip10pt}
    \renewcommand{\bibname}{DAFTAR PUSTAKA}
    \addcontentsline{toc}{chapter}{DAFTAR PUSTAKA}
    %\bibliographystyle{unsrt} % untuk style dengan nomor
    \bibliographystyle{format-pustaka} % untuk style dengan nama
    \bibliography{pustaka}
    
%---------------------------------------------------------------------
%   APPENDIX
%---------------------------------------------------------------------

    {
    \appendix
    \addtocontents{toc}{\vskip10pt}
    \renewcommand{\chaptername}{LAMPIRAN}
    %%%%%%%%%%%%%%%%%%%%%%%%%%%%%%%%%%%%%%%%%%%%%%%%%%%%%%%%%%%%%%%%%%%%%%
% LAMPIRAN A
%%%%%%%%%%%%%%%%%%%%%%%%%%%%%%%%%%%%%%%%%%%%%%%%%%%%%%%%%%%%%%%%%%%%%%

\chapter{Judul Lampiran}

%---------------------------------------------------------------------
\section{Lampiran}
%---------------------------------------------------------------------

Tuliskan lampiran di sini.

%%%%%%%%%%%%%%%%%%%%%%%%%%%%%%%%%%%%%%%%%%%%%%%%%%%%%%%%%%%%%%%%%%%%%%
    %\input{./halaman-belakang/lampiran/lampiranB}
    }
%---------------------------------------------------------------------
%   BIOGRAFI PENULIS
%---------------------------------------------------------------------

    %%%%%%%%%%%%%%%%%%%%%%%%%%%%%%%%%%%%%%%%%%%%%%%%%%%%%%%%%%%%%%%%%%%%%%
%%%%%%%%%%%%%%%%%%%%%%%%%%%%%%%%%%%%%%%%%%%%%%%%%%%%%%%%%%%%%%%%%%%%%%
% BIOGRAFI PENULIS
%=====================================================================
\renewcommand{\thechapter}{\Roman{chapter}}
\addtocontents{toc}{\vskip10pt}
\chapter*{BIOGRAFI PENULIS}
\renewcommand{\thechapter}{\arabic{chapter}}
%---------------------------------------------------------------------

\begin{wrapfigure}{l}{.35\textwidth}
  \begin{center}
    \vspace{-20pt} % Silahkan disesuaikan apabila space di atas gambar terlalu kecil/besar
    \includegraphics[width=.30\textwidth]{./gambar/foto.png}
    \vspace{-20pt} % Silahkan disesuaikan apabila space di bawah gambar terlalu kecil/besar
  \end{center}
\end{wrapfigure}

\noindent Biodata singkat mengenai penulis dapat dituliskan pada bagian ini. Gambar diri dari \textit{file} "foto.jpeg" dalam folder  \textbf{gambar} akan ditampilkan di bagian atas kalimat ini. \lipsum[1-2]
 
\vspace{7pt}
\noindent Email: \emailMahasiswa

%======================================================================
    
%=====================================================================
\end{document}
%=====================================================================

%%%%%%%%%%%%%%%%%%%%%%%%%%%%%%%%%%%%%%%%%%%%%%%%%%%%%%%%%%%%%%%%%%%%%%