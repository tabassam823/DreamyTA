%%%%%%%%%%%%%%%%%%%%%%%%%%%%%%%%%%%%%%%%%%%%%%%%%%%%%%%%%%%%%%%%%%%%%%
% KODE RINGKAS UNTUK SIMBOL DAN SATUAN:
%=====================================================================
% Simbol Matematis
%---------------------------------------------------------------------

\def\imag{\mathrm{i}}
\def\euler{\mathrm{e}}
\def\diff{\mathrm{d}}

%---------------------------------------------------------------------
% Satuan
%---------------------------------------------------------------------

\def\unitangstrom{\,\textrm{\AA}}
\def\unitkg{\,\textrm{kg}}
\def\unitJ{\,\textrm{J}}
\def\unitev{\,\textrm{eV}}
\def\unitmev{\,\textrm{meV}}
\def\unitvolt{\,\textrm{V}}
\def\unitm{\,\textrm{m}}
\def\unitcm{\,\textrm{cm}}
\def\unitmm{\,\textrm{mm}}
\def\unitum{\,\mathrm{\mu}\textrm{m}}
\def\unitnm{\,\textrm{nm}}
\def\unitwn{\,\textrm{cm}^{-1}}
\def\unitsec{\,\textrm{s}}
\def\unitps{\,\textrm{ps}}
\def\unitfs{\,\textrm{fs}}
\def\unitdeg{^\circ}
\def\unitcelcius{\unitdeg\textrm{C}}
\def\unitpercent{\,\%}

%---------------------------------------------------------------------
% Singkatan
%---------------------------------------------------------------------

\def\its{\,Institut Teknologi Sepuluh Nopember}
\def\qe{\,\textsc{Quantum\:ESPRESSO}}
\def\btp{\,\textsc{BoltzTraP2}}
\def\schro{\,Schr\"{o}dinger}
\def\snse{\,SnSe}

%%%%%%%%%%%%%%%%%%%%%%%%%%%%%%%%%%%%%%%%%%%%%%%%%%%%%%%%%%%%%%%%%%%%%%

