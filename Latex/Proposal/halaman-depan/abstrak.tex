%%%%%%%%%%%%%%%%%%%%%%%%%%%%%%%%%%%%%%%%%%%%%%%%%%%%%%%%%%%%%%%%%%%%%%
%
%   Abstrak
%
%%%%%%%%%%%%%%%%%%%%%%%%%%%%%%%%%%%%%%%%%%%%%%%%%%%%%%%%%%%%%%%%%%%%%%

\begin{center}
    \addcontentsline{toc}{chapter}{ABSTRAK}
    \pagestyle{fancy}
\end{center}

%---------------------------------------------------------------------

\begin{center}
    {\textbf{\MakeUppercase{\judulTA}}}
\end{center}

\vspace{5mm}

\noindent \begin{tabular}{l c l}
    \textbf{Nama}       & \textbf{:} & \textbf{\namaMahasiswa}  \\[-1mm]
    \textbf{NRP}        & \textbf{:} & \textbf{\noIndukMahasiswa}  \\[-1mm]
    \textbf{Departemen} & \textbf{:} & \textbf{\namaDepartemen}  \\[-1mm]
    \textbf{Pembimbing} & \textbf{:} & \textbf{1. \namaDosenPembimbingSatu}  \\[-1mm]
                        &            & \textbf{2. \namaDosenPembimbingDua}
\end{tabular}

%---------------------------------------------------------------------

\vspace{5mm}

\begin{center}
    \noindent {\textbf{{Abstrak}}}
\end{center}

%---------------------------------------------------------------------

% Catatan: Gunakan \singlespacing di tiap awal paragraf

{\singlespacing\indent%
Ini adalah contoh dokumen Tugas Akhir yang dibuat dengan menggunakan \textit{template} \LaTeX{} dengan format yang telah disesuaikan dengan aturan penulisan Tugas Akhir yang berlaku di Departemen Fisika, \its{} (ITS). \textit{Template} ini dibuat dengan tujuan untuk memudahkan mahasiswa/(i) dalam melakukan penyusunan Tugas Akhir sekaligus untuk dapat digunakan sebagai \textit{template} yang berlaku umum, dengan beberapa penyesuaian untuk Tugas Akhir di Departemen lain di ITS. \textit{File} Tugas Akhir dalam format \texttt{$^*$.pdf} akan dapat dihasilkan dengan mengkompilasi \texttt{main.tex} menggunakan \textit{compiler} Lua\LaTeX.
}

%---------------------------------------------------------------------

\vspace{5mm}

\noindent \textbf{Kata kunci:} \textit{katakunci-1, katakunci-2, katakunci-3, katakunci-4} % Kata kunci dalam bahasa Indonesia

%%%%%%%%%%%%%%%%%%%%%%%%%%%%%%%%%%%%%%%%%%%%%%%%%%%%%%%%%%%%%%%%%%%%%%